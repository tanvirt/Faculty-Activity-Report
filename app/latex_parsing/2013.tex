\documentclass{article}

%\usepackage[innermargin=-2.0cm,]{fullwidth}
\usepackage{fullpage}
\usepackage{color}
\usepackage{url}

\newcommand{\fillin}{ {\color{red} NOT FINISHED } }

\begin{document}

\title{COLLEGE OF ENGINEERING \\ Annual Activities Report}
\date{}
\maketitle

\section{NAME:} \begin{Large}Alin V. Dobra\end{Large}

\section{TENURE:} Tenured

\section{CURRENT RANK:} Associate Professor, Computer and Information Science and Engineering

\section{DATE APPOINTED TO THIS RANK:} August 2009

\section{AFFILIATE APPOINTMENTS:} None

%\setcounter{section}{1}
%\setcounter{page}{2}

%% \section{DESCRIPTION OF JOB DUTIES}

%% As an assistant professor, my job duties consist in conducting
%% scholarly research , graduate and undergraduate teaching and
%% professional service.

%% \section{AREA OF SPECIALIZATION}

%% Databases, Approximate Query Processing, Data-mining.

\section{ASSIGNED ACTIVITY}

\begin{center}
\begin{tabular}{lccc}
& Spring 2013 & Summer 2013 & Fall 2013 \\
\hline
Teaching & 45\% & 2\% & 50\% \\
Research & 40\% & 90\% & 35\% \\
Service & 15\% & 8\% & 15\%  \\
\hline
Total & 100\% & 100\% & 100\%
\end{tabular}
\end{center}

%% \section{EDUCATIONAL BACKGROUND}

%% \begin{center}
%%   \begin{tabular}{lccc}
%%     University & Field of study & Degree & Date Awarded \\
%%     \hline
%%     Cornell University & Computer Science & Ph.D & Aug. 2003 \\ 
%%     Cornell University & Computer Science & MS* & Dec. 2001 \\
%%     Technical University of Cluj-Napoca & Computer Science & BS & July
%%     1998 
%%   \end{tabular}
%% \end{center}

%% (*) Cornell University awards an MS to the Ph.D students when they
%% pass the thesis proposal phase. I was not enrolled in a
%% separate MS program at Cornell U.

%% \section{EMPLOYMENT}

%% \begin{center}
%%   \begin{tabular}{lcc}
%%     Employer & Rank & Period \\ 
%%     \hline
%%     University of Florida & Assistant Professor (tenure-accruing) &
%%     2003--present \\
%%     Cornell University & Research and Teaching Assistant & 1998--2003
%%   \end{tabular}
%% \end{center}

%% \section{YEAR TENURE/PERMANENT STATUS WAS AWARDED BY UNIVERSITY OF
%%     FLORIDA}

%% N/A

%% \section{TENURE \& PROMOTION CRITERIA}

%% \subsection{University of Florida - Tenure and/or Promotion Process - 2008-2009}

%% \begin{enumerate}
%%   \item The university’s criteria for granting tenure, promotion, or
%%     permanent status shall be relevant to the performance of the work
%%     that the faculty member has been employed to do and to his/her
%%     performance of the duties and responsibilities expected of a
%%     member of the university community. These criteria recognize three
%%     broad categories of academic service as follows:

%%     \begin{enumerate}
%%     \item Teaching –Instruction, including regular classroom teaching
%%       and distance/executive/continuing education, direction of theses
%%       and dissertations, academic advisement, extension education
%%       programs, and all preparation for this work, including study to
%%       keep abreast of one’s field.
      
%%     \item Research – Research or other creative activity including, peer-reviewed
%%       publications.
      
%%     \item Service – Public and professional.
      
%%       Changes in tenure and promotion criteria, including the
%%       discipline-specific departmental clarifications of those criteria,
%%       shall not become effective until one year following adoption of the
%%       changes.
%%     \end{enumerate}

%%   \item In most cases, tenure and promotion require distinction in at
%%     least two areas, one of which shall be that of the faculty member’
%%     s primary responsibility, and those areas should be teaching and
%%     research unless the faculty member or extension faculty member has
%%     an assignment that primarily reflects other responsibilities, such
%%     as the Cooperative Extension Service. Merit should be regarded as
%%     more important than variety of activity. “Distinction” in the
%%     categories is defined by the University and clarified by each
%%     college, and department in terms tailored to the college and to
%%     department disciplines and consistent with University
%%     standards. Faculty and new hires should receive a copy of the
%%     college’s or equivalent unit’s and department’s criteria
%%     clarifying the expectations for promotion and tenure and the
%%     definition of distinction.
%% \end{enumerate}

%% \subsection{College of Engineering Tenure and/or Promotion Criteria for Faculty – 2008-09}


%% The College of Engineering criteria statement is as follows:
%% \begin{enumerate}

%% \item As a major unit of the University of Florida, the College of
%%   Engineering pursues the same mission as the university, and promotes
%%   excellence in teaching, research, and service.

%% \item Evaluation of tenure-track faculty members in the Professorial
%%   Series for promotion, tenure and salary adjustment focuses on
%%   performance in teaching, research, and service. To be recommended
%%   for promotion, tenure, or merit salary increases, a faculty member
%%   is expected to have an outstanding record in two of these
%%   areas. Since the principal responsibilities of each department are
%%   teaching and research, performance in these areas is emphasized
%%   unless the candidate’s service contributions are extraordinary in
%%   significance, impact and visibility. Service to the public school
%%   sector is considered to be important and will be considered in the
%%   evaluation process.

%% \item For tenure and promotion to Associate Professor, teacher
%%   evaluations, success in securing funded research, publications in
%%   scholarly journals, honors and awards, national recognition,
%%   Ph.D. production, and potential for long term success will be taken
%%   into consideration. For promotion to Professor, the candidate must
%%   have established a distinguished record in his/her field with
%%   evidence of national and international recognition. He/she must have
%%   excelled in teaching and research and have an impressive record of
%%   service to the profession at both national and international
%%   levels. The quality as well as the quantity of technical
%%   contributions will be judged.
%% \end{enumerate}

%% \subsection{Computer Information Science \& Engineering Department
%%   Tenure and/or Promotion Criteria}

%%  Discipline specific criteria not yet adopted.



\section{TEACHING, ADVISING AND/OR INSTRUCTIONAL ACCOMPLISHMENTS}

\textbf{Teaching Philosophy:} My main strength as a teacher, and a 
real asset as a researcher, is an broad understanding of Computer
Science as a whole rather than just a few subjects, as it is often the
case. This breadth of knowledge allows me to easily teach a variety of
subjects well outside my research area. 

\begin{itemize}
\item \textbf{COP 5615 Distributed Operating System Principles} This
  is a graduate level course focused on distributed systems and modern
  applications like large ecommerce sites, peer-to-peer systems, and
  distributed computation. My overall instruction evaluation for this class was 4.49 out of 5.0
\item \textbf{COP 6726 Database System Implementation} This is a very
  advanced class on database system implementation. Students are
  required to implement a full fledged database that can execute SQL
  and were taught classic and modern takes on how to implement high
  performance databases.  My overall instruction evaluation for this class was 4.48 out of 5.0
\end{itemize}

\textbf{Supervising:} I supervised 3 Ph. D students during the last 1
year.  The active students are Supriya Nirkhiwale and Andrei
Todor. Andrei is co-advised with Tamer Kahveci. Louis Cheung, my 3rd
student decided to abandon the Ph-D program.

\section{TEACHING EVALUATIONS}

\noindent Course: COP 5615 \hfill Required: NO \\
\noindent Fall, 2013 \hfill \# Enrolled: 152\\
High=5, Low=1 \hfill \# Responses: 37
\nopagebreak[4]
\begin{center}
  \begin{tabular}{llll}
    Question & Dr. Dobra & Dept. & College \\
    & Mean & Mean & Mean \\
    \hline
    1. Description of course objectives and assignments & 4.41 & 4.12 & 4.24 \\
    2. Communication of ideas and information & 4.35	&3.97	&4.05 \\
    3. Expression of expectations for performance in this class 
    & 4.49	&4.07	&4.17\\
    4. Availability to assist students in or out of class & 4.43	&4.18	&4.19\\
    5. Respect and concern for students & 4.49	&4.29	&4.32\\
    6. Stimulation of interest in course & 4.57	&4.04	&4.08 \\
    7. Facilitation of learning & 4.49	&4.00	&4.03 \\
    8. Enthusiasm for the subject & 4.73	&4.32	&4.38 \\
    9. Encouragement of independent, creative, and critical thinking & 4.76	&4.19	&4.18 \\
    \hline
    10. Instructor Overall & 4.49	&4.09	&4.17
  \end{tabular}
\end{center}

\noindent Course: COP 6726 \hfill Required: NO \\
\noindent Fall, 2012 \hfill \# Enrolled: 112\\
High=5, Low=1 \hfill \# Responses: 48
\nopagebreak[4]
\begin{center}
  \begin{tabular}{llll}
    Question & Dr. Dobra & Dept. & College \\
    & Mean & Mean & Mean \\
    \hline
    1. Description of course objectives and assignments & 4.29	&4.06	&4.15 \\
    2. Communication of ideas and information & 4.46	&3.94	&4.01 \\
    3. Expression of expectations for performance in this class  & 4.42	&4.04	&4.13\\
    4. Availability to assist students in or out of class & 4.31 &4.13	&4.14 \\
    5. Respect and concern for students & 4.52	&4.22	&4.26 \\
    6. Stimulation of interest in course & 4.68	&4.01	&4.07 \\
    7. Facilitation of learning & 4.40 & 3.96&	4.03 \\
    8. Enthusiasm for the subject & 4.70 &4.31	&4.37 \\
    9. Encouragement of independent, creative, and critical thinking & 4.69&	4.18&	4.20\\
    \hline
    10. Instructor Overall & 4.48 &  4.05 & 4.15 \\
  \end{tabular}
\end{center}


\section{GRADUATE COMMITTEE ACTIVITIES}

So far, I have served in 70 committees (9 as Chair, 3 as Co-Chair, 1 External Member, 55 as Member and 2 as Minor).

\begin{center}
  \begin{tabular}{llllll}
    Role & Student Name & Degree & Major & Degree Date\\
    \hline
Chair&ALEX,DALEY&M.S.&Computer Engineering&8/14/2012 \\
Chair&CHEN,LIXIA&Ph.D.&Computer Engineering&12/20/2011\\
Chair&CHEUNG,LOUIS&Ph.D.&Computer Engineering&\\
Chair&DHURANDHAR,AMIT S&Ph.D.&Computer Engineering&8/11/2009\\
Chair&GOLANI,GURUDITTA&M.S.&Computer Sciences&8/6/2005\\
Chair&JAMPANI,RAVINDRANATH&Ph.D.&Computer Engineering&8/14/2012\\
Chair&KIM,EUNKEE&Ph.D.&Computer Engineering& \\
Chair&NIRKHIWALE,SUPRIYA&Ph.D.&Computer Engineering& \\
Chair&RUSU,FLORIN I&Ph.D.&Computer Engineering&5/5/2009 \\
Chair&TODOR,ANDREI&Ph.D.&Computer Engineering&\\
Chair&ZHENG,YONGJIE&Ph.D.&Computer Engineering&\\
Co-Chair&CHITNIS,LAUKIK VILAS&Ph.D.&Computer Engineering&8/12/2008\\
Co-Chair&MARTINEZ,ALEXANDRA MARIA&Ph.D.&Computer Engineering&12/18/2007\\
External&YAO,KAI&Ph.D.&Materials Sc. and Eng.&\\
Member&ABBASIMOGHADDAM,SAEED&Ph.D.&Computer Engineering&12/18/2012\\
Member&ARUMUGAM,SUBRAMANIAN&Ph.D.&Computer Engineering&8/12/2008\\
Member&AY,FERHAT&Ph.D.&Computer Engineering&8/9/2011\\
Member&BAKER,TROY A&Ph.D.&Computer Engineering&\\
Member&BANDYOPADHYAY,NIRMALYA&Ph.D.&Computer Engineering&12/20/2011\\
Member&CHEN,TAO&Ph.D.&Computer Engineering&8/14/2012\\
Member&CHEN,YANG&Ph.D.&Computer Engineering&\\
Member&CHOU,HONGCHI&Ph.D.&Computer Engineering&\\
Member&EOM,BOYUN&M.S.&Computer Sciences&8/6/2005\\
Member&GABR,HAITHAM MOHAMMAD&Ph.D.&Computer Engineering&\\
Member&GRANT,CHRISTAN EARL&Ph.D.&Computer Engineering&\\
Member&GULSOY,GUNHAN&Ph.D.&Computer Engineering&8/13/2013\\
Member&HAN,SEUNG CHUL&Ph.D.&Computer Engineering&5/5/2007\\
Member&HSU,WEI-JEN&Ph.D.&Computer Engineering&8/12/2008\\
Member&JOSEPH,REJITH G&M.S.&Computer Engineering&5/3/2011\\
Member&JOSHI,SHANTANU SHARAD&Ph.D.&Computer Engineering&8/14/2007\\
Member&KANJILAL,VIRUPAKSHA&Ph.D.&Computer Engineering&12/18/2012\\
Member&KHAN,MD ARIFUL&M.S.&Computer Engineering&8/10/2010\\
Member&KHANAPURE,VISHAL A&M.S.&Computer Engineering&8/11/2009\\
Member&KIM,EUNJU&Ph.D.&Computer Engineering&5/7/2013\\
Member&KUMAR,UDAYAN&Ph.D.&Computer Engineering&12/18/2012\\
Member&LI,BO&Ph.D.&Computer Engineering&8/14/2012\\
Member&LI,KUN&Ph.D.&Computer Engineering&\\
Member&LI,XUEHUI&Ph.D.&Computer Engineering&12/18/2007\\
Member&LIU,HECHEN&Ph.D.&Computer Engineering&12/18/2012\\
Member&LIU,JUN&Ph.D.&Computer Engineering&5/6/2008\\
Member&LU,I-HSUAN&Ph.D.&Computer Engineering&\\
Member&MCKENNEY,MARK A&Ph.D.&Computer Engineering&8/12/2008\\
Member&MOOLA,ANIL&M.S.&Computer Engineering&5/6/2008\\
Member&MUN,MIN YOUNG&Ph.D.&Computer Engineering&\\
Member&PANSARE,NIKETAN R&M.S.&Computer Engineering&12/22/2009\\
Member&PAULY,ALEJANDRO&Ph.D.&Computer Engineering&5/5/2007\\
Member&PEREZ,LUIS L&Ph.D.&Computer Engineering&\\
Member&PEREZ,LUIS L&M.S.&Computer Engineering&12/22/2009\\
Member&POL,ABHIJIT A&Ph.D.&Computer Engineering&8/14/2007\\
Member&PRAING,REASEY&Ph.D.&Computer Engineering&8/12/2008
  \end{tabular}
\end{center}


\begin{center}
  \begin{tabular}{llllll}
    Role & Student Name & Degree & Major & Degree Date\\
    \hline
Member&RAJAMANICKAM,SIVASANK&Ph.D.&Computer Engineering&12/22/2009\\
Member&RASHEED,HASSAN S&Ph.D.&Computer Engineering&5/5/2009\\
Member&RAVUNNIKUTTY,GIRISH&M.S.&Computer Engineering&5/3/2011\\
Member&SEN,SOMAK&M.S.&Computer Engineering&5/5/2009\\
Member&SHRIVASTAVA,KARTIK P&M.S.&Computer Engineering&8/10/2010\\
Member&SOMAIYA,MANAS H&Ph.D.&Computer Engineering&12/22/2009\\
Member&THAKUR,GAUTAM&Ph.D.&Computer Engineering&12/18/2012\\
Member&VENKATESWARAN,JAYENDRA G&Ph.D.&Computer Engineering&12/18/2007\\
Member&VISWANATHAN,GANESH&Ph.D.&Computer Engineering&12/20/2011\\
Member&WANG,YIBIN&Ph.D.&Computer Engineering&\\
Member&WEINRICH,BRIAN ERWIN&Ph.D.&Computer Engineering&5/5/2007\\
Member&WU,MINGXI&Ph.D.&Computer Engineering&8/12/2008\\
Member&XU,FEI&Ph.D.&Computer Engineering&8/11/2009\\
Member&YERALAN,SENCER NURI&Ph.D.&Computer Engineering&\\
Member&YUAN,WENJIE&Ph.D.&Computer Engineering&8/9/2011\\
Member&YUN,YOUNGSANG&Ph.D.&Computer Engineering&8/10/2010\\
Member&ZANDI,HELIA&Ph.D.&Computer Engineering&\\
Member&ZENG,QI&Ph.D.&Computer Engineering&\\
Member&ZHANG,XU&Ph.D.&Computer Engineering&12/18/2007\\
Minor&VEERAMANI,KARTHIK&M.S.&Electrical and Computer Eng.&12/16/2006\\
Minor&VELUCHAMY,NIVETHA&M.S.&Electrical and Computer Eng.&5/4/2010
  \end{tabular}
\end{center}

\section{CONTRIBUTION TO DISCIPLINE/RESEARCH NARRATIVE}

With the advent of cheap storage and fast networks, data is produced
and stored at high speeds. The next big challenge in computer science
is to process efficiently these large volumes of data.  My research
goal is to develop mathematical methods that allow design and analysis
of algorithms that can process such large volumes of data, on one
hand, and deeper understanding of existing algorithms on the other
hand. Specific contributions that are the subject of my work are:
\begin{itemize}
\item Processing of data arriving at high speeds in large volumes, for
  example for data generated by computer networks, depends crucially
  on the ability to accurately summarize the data \emph{on the fly}. I
  am developing techniques for producing provably good summarization
  and mathematical infrastructure that speeds up the design of such
  techniques. My work in this area is supported by an NSF CAREER
  award.
\item Large databases are prevalent nowadays -- Walmart records and
  stores every customer and warehouse transaction. Processing queries
  for such large databases is problematic since techniques from
  traditional database research lead to large running times for the
  queries. An alternative is to process the queries on a sample from
  the database and to estimate the result -- while this idea is simple
  in principle, it leads to complicated analysis and a complete
  redesign of query processing engines. My work in this direction,
  together with Dr. Chris Jermaine, was published in SIGMOD 2005,06,07,08 and
  VLDB 2005 conferences. It is also supported by an NSF grant for the
  next 4 years.

  An interesting new directions this work lead to is the development of
  very fast databases. The result with Chris Jermain, published in
  SIGMOD 2010, rivals existing database solutions on data in the 10 TB
  range for a fraction of the cost -- a 40,000\$ computer running
  DataPath rivals a 10 million \$ IBM DB2 solution for certain
  analytical queries.

\end{itemize}

Most of my work is published in highly competitive conferences (with
low acceptance rate) and journals. A number of my papers had a
significant impact in the community, as reflected by the number of
citations they received:
\begin{itemize}
  \item \emph{Gossip-based computation of aggregate information}: 937
    citations
  \item \emph{Processing complex aggregate queries over data streams}:
    330 citations
  \item \emph{SECRET: a scalable linear regression tree algorithm}: 72
    citations
  \item \emph{Sketch-based multi-query processing over data streams}: 46 citations
\end{itemize}

I have 1824 citations overall, an h-index of 14 and i10-index of 21. 

(\texttt{http://scholar.google.com/citations?user=smEOOBsAAAAJ})


\section{CREATIVE WORKS OR ACTIVITIES}

I developed/help develop three major software systems:
\begin{itemize}
\item BioVerto: a tool for visualizing and analyzing biological networks 
(\texttt{http://bioverto.org}). January 2013 (joint effort with Tamer Kahveci)
\item DBO: a database system that estimates the query result while the
  query is processed. This system is developed in collaboration with
  Chris Jermaine. 2006-2013
\item DataPath: a fast database system particularly suited for large
  analytical query loads. This system is developed in collaboration with
  Chris Jermaine. 2009-now
\item GLADE: an addon on top of DataPath that allows sophisticated
  data processing as part of a relational database engine. Collaboration with Sanjay Ranka, 2011-now
\end{itemize}

\section{PATENTS AND COPYRIGHTS}

\begin{enumerate}
  \item \emph{Sketch-based multi-query processing over data streams}, 
   Alin Dobra, Johannes Gehrke, Rajeev Rastogi, Minos Garofalakis,
   Patent number: 7328220, Issue date: Feb 5, 2008

   This patent covers the innovations in terms of extending a
   particular approximation technique, sketching, to multiple queries
   that are run simultaneously. The main innovation is a summary 
   sharing technique that is useful for a wide class of approximations.

\end{enumerate}

\section{PUBLICATIONS}

\begin{enumerate}
  \renewcommand{\labelenumi}{\alph{enumi}}
  \item[a.] \textbf{Books, Sole Author} - N/A
  \item[b.] \textbf{Books, Co-authored} - N/A
  \item[c.] \textbf{Books, Edited} - N/A
  \item[d.] \textbf{Books, Contributor} 
  \item[e.] \textbf{Monographs} - N/A
  \item[f.] \textbf{Refereed Publications}
    
    \textbf{Journal Publications} 

    \begin{enumerate}
      \renewcommand{\labelenumi}{\arabic{enumii}}
      \item[1.] Andrei Todor, Haitham Gabr, Alin Dobra, Tamer Kahveci. Large scale analysis of signal reachability. Bioinformatics. to appear
      \item[2.] Andrei Todor, Alin Dobra, Tamer Kahveci. Characterizing topology of probabilistic biological networks. IEEE/ACM Transactions on Computational Biology and Bioinformatics Journal (IEEE/ACM TCBB). 10:4. 2013
      \item[3.] Andrei Todor,  Alin Dobra, Tamer Kahveci. Probabilistic Biological Network Alignment. IEEE/ACM Transactions on Computational Biology and Bioinformatics Journal (IEEE/ACM TCBB). 10:1, 2013
      \item[4.] Supriya Nirkhiwale, Alin Dobra, Christopher M. Jermaine: A Sampling Algebra for Aggregate Estimation. PVLDB 6(14): 1798-1809 (2013)
    \end{enumerate}

    \textbf{Refereed Proceedings}
    \begin{enumerate}
      \item[1.] Haitham Gabr, Andrei Todor, Helia Zandi, Alin Dobra, Tamer Kahveci. PReach: Reachability in Probabilistic Signaling Networks. International Conference on Bioinformatics and Computational Biology (ACM-BCB), 2013
    \end{enumerate}
\end{enumerate}

\section{LECTURES, SPEECHES OR POSTERS PRESENTED AT PROFESSIONAL
  CONFERENCES/MEETINGS}


\subsection*{International}

N/A

\subsection*{State}

N/A

\subsection*{Local}
N/A

\section{CONTRACTS AND GRANTS SINCE LAST PROMOTION OR DURING THE LAST
  FIVE YEARS}

\subsection*{Funded Research Contracts and Grants}

\begin{enumerate}
\item[a.] \textbf{Funded Externally}

\begin{center}
  \begin{tabular}{lllllll}
    Title of Grant & Funding & PI & Start-End & Value
    &Funding\\
    & Agen. & & & &  Portion \\
    \hline
    IIS: EAGER: A Framework for Large Data & NSF & PI & 09/01/11-/8/31/14/ & \$100,000 & \$100,000\\
    Analysys & & & & \\
    CIF: EAGER: Modeling Prob Bio Networks & NDF & Co-PI & 08/13-07/15 & \$175,000 & \$87,500 \\

  \end{tabular}
\end{center}

\textbf{
  \begin{center}
    Summary of Grant Funding Received, Mar 2013-present
    \begin{tabular}{lc}
      Role & Total \\
      \hline
      Principal Investigator & \$100,000 \\
      Co-Principal Investigator & \$87,500 \\
      \hline
      Totals & \$187,500
    \end{tabular}
  \end{center}
}

\item[b.] \textbf{Funded Internally}
N/A

\item[c.] \textbf{Submitted, Pending Decision}

N/A

\item[d.] \textbf{Submitted Not Funded}

\begin{center}
  \begin{tabular}{llllll}
    Title of Grant & Fund.  & PI & Start-End & Value \\
    & Agen. & & &  Funding \\
    \hline
    BIGDATA: Small: DA: GLADE: A Framework for.. & NSF & Co-PI & 01/01/13-12/31/15 & \$750,000 \\
    II-NEW: Research infrastructure for Cost Effective.. & NSF & Co-PI & 04/01/13-03/31/15 & \$600,022\\
  \end{tabular}
\end{center}

\end{enumerate}

\section{UNIVERSITY GOVERNANCE AND SERVICE}

\subsection{Department Committee Memberships}
\begin{itemize}
  \item Colloquium Committee 
  \item Chair of Scholarships and Awards Committee 
\end{itemize}

\section{CONSULTATIONS OUTSIDE THE UNIVERSITY}

N/A

\section{EDITOR OF A SCHOLARLY JOURNAL, SERVICE ON AN EDITORIAL
  ADVISORY BOARD OR REVIEWER FOR A SCHOLARLY JOURNAL}

Reviewer for the following journals:

\begin{itemize}
\item IEEE Transactions on Parallel and Distributed Computing
\end{itemize}

\section{INTERNATIONAL ACTIVITIES}

N/A

\section{SERVICE TO SCHOOLS}

N/A

\section{MEMBERSHIP AND ACTIVITIES IN THE PROFESSION}
\begin{itemize}
\item \emph{Program committee} member for the following conferences:
  \begin{itemize}
  \item International Conference on Data Engineering, 2014
  \item ACM-SIGMOD 2014
  \end{itemize}
\end{itemize}

\section{HONORS}

%\thispagestyle{empty}
%\setcounter{section}{}

\section{THE FURTHER INFORMATION SECTION}

\vspace{2in}


\begin{center}
Signature\line(1,0){200}\hspace{5em} Date\line(1,0){50}
\end{center}

\end{document}
